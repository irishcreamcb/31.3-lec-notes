\chapter{Why you can't ``Add to Infinity'', and why we do it anyway}
When I first encountered infinite series in Kumon, of all places, it was my least favorite part of math (one, of course, of many; I used to \textit{hate} math). 
It was just so counterintuitive. How the flip do you add an infinite amount of numbers? Then, more than 5 years later in the year of our Lord 2024, I read 
Jay Cummings' Real Analysis textbook and understood what it meant for a series to converge to a value---which is what we mean when we write \(\sum_{n=0}^{\infty} f(n) = s\), 
for some constant \(s\)---and it all made sense. Actual convergence won't be tackled until MATH 90.1, as far as I know, so you needn't worry yourself with it now, but if
you're comfortable with \(\varepsilon-\delta\) limits for functions as \(x \to \infty\), the idea is very similar.\par 
Anyway, my point is that ``adding to infinity'' is rightfully nonsense, and understanding sequences and series becomes much more rewarding once you do away 
with that idea. Section 1 covers some elementary properties about series' little siblings, \textbf{sequences}, before we get into actual \textbf{series} in section 2. 
Then, section 3 covers the various \textbf{tests for convergence}, of which there are many (oh, so many). Finally, section 4 talks about representing \textbf{functions as power series},
primarily focusing on the infamous Taylor series expansions.
\section{Sequences} 
\section{Series}
\section{Tests for convergence}
\section{Functions as power series}