\chapter{Calculus I Addendums}
I've labelled this section as such because it really feels like these topics should have been covered in MATH 31.1 and 31.2, seeing as they're focused on
limits, derivatives, and integrals---all the key aspects of basic calculus---but they're just tacked on here for some reason. Regardless, here they are.
Section 1 focuses on \textbf{solving limits by using derivatives}, and section 2 focuses on \textbf{integrals of functions on infinite intervals}, \textit{or}
\textbf{functions that diverge to infinity on closed intervals.}  

\section{Indeterminate forms and \lh's rule} 
Recall the following limit from elementary calculus:\[
    \lim_{x\to 0 } \frac{\sin x}{x} = 1. 
\] You might remember that this limit was proved using arc lengths of a circle, or maybe the squeeze theorem. Either way, it's clear that we can't use
the simpler methods of solving a limit (like the limit laws) to find this one. Direct substitution yields the fraction \(\frac{0}{0}\), which isn't very helpful.\par 
Now, imagine you're Johann Bernoulli looking at this equation in the 17th century and you notice a few things:\begin{enumerate}
    \item that \(\sin x\) and \(x\) are both differentiable, with \(
        \dfrac{\der }{\der x} \sin x = \cos x \) and \( \dfrac{\der }{\der x} x = 1 
    \) respectively,
    \item that both \(f' = \cos x\) and \(g' = 1\) are continuous throughout the real numbers, and 
    \item the limit of \(\dfrac{\cos x}{1}\) as \(x\) approaches \(1\) is, trivially, 1. 
\end{enumerate} 
Huh! Isn't that interesting! Being the mathematician that you are, you immediately think of whether or not this can be generalized to include a broader range 
of functions, and not just relatively simple ones like \(\sin x\) and \(x\).\par 
You decide to investigate the fact that both functions were equal to 0 when evaluated at \(x = a\), so you set up two functions \(f\) and \(g\), where \(f(a) = g(a) = 0\). 
For good measure (and for our convenience), you assume that both their derivatives are continuous and that \(g'(x) \neq 0\). Then, you try evaluating the limit of their quotient, as such:\begin{align*}
    \lim_{x\to a} \frac{f(x)}{g(x)} &= \lim_{x\to a} \frac{f(x) - f(a)}{g(x) - g(a)},~\text{since \(f(a) = g(a) = 0\).}\\
    \intertext{Then, we can divide the numerator and denominator by the same value, \(x - a\), without changing the overall value:} 
    &= \lim_{x \to a} \frac{\dfrac{f(x) - f(a)}{x-a}}{\dfrac{g(x) - g(a)}{x - a}} = \frac{\displaystyle{} \lim_{x\to a} \frac{f(x) - f(a)}{x-a}}{\displaystyle{} \lim_{x\to a} \frac{g(x) - g(a)}{x-a}}\\ 
    \intertext{Wait a minute. That looks familiar! These limits on the numerator and the denominator are precisely \(f'(a)\) and \(g'(a)\)!}
    &= \frac{f'(a)}{g'(a)} = \lim_{x \to a} \frac{f'(x)}{g'(x)},~\text{since \(f'\) and \(g'\) are continuous.} 
\end{align*}
Of course, you being one of the smartest mathematicians of all time, you don't stop there. You generalize it even further to go beyond the rather restrictive conditions 
we placed on ourselves for this example to work as nicely as it did, armed with all the tools of your calculus-addled brain.\footnote{Proving the very generalized and 
extended \lh's rule involves a lot of calculus that we haven't learned yet. I might include a proof of it in the latter pages or something, but there are \textit{a lot} of cases.} 
Satisfied with yourself, but desperate for some money, you sell it to a colleague, who promptly publishes it, and the theorem is now forever named after him and not after you. 
Oh well. At least you have about a dozen generations left to solidify your family name's fame.\par 
We now introduced the generalized theorem to solve limits which take on the form \(\frac{0}{0}\) or \(\frac{\pm \infty}{\pm \infty}\) when direct substitution is attempted, named after
17th century mathematician Guillaume de \lh.
\begin{theo}[\lh's rule]
Suppose that\begin{itemize}
    \item two functions \(f\) and \(g\) are differentiable;
    \item \(g'(x) \neq 0 \) on some open interval \(I\) that contains a constant \(a\), except
possibly \(g'(a) = 0\); and 
    \item \(\lim_{x\to a} f(x) = \lim_{x\to a} g(x) = 0\), or both limits diverge to either positive or negative infinity.
\end{itemize} Then,\[
    \lim_{x\to a} \frac{f(x)}{g(x)} = \lim_{x\to a} \frac{f'(x)}{g'(x)}, 
\] provided that the limit on the right-hand side of the equation exists. 
\end{theo}
Note that the constant \(a\) that \(x\) approaches in this limit may be either from the right or the left (i.e.\ \(a^+\) or \(a^-\)), or could also be as \(x\) 
approaches either positive or negative infinity.\par 
\begin{example}
    Evaluate \(\displaystyle{} \lim_{x\to -1} \frac{\sin \,(x^3 + 1)}{x^2 -1 }\).\par 
    Notice here that when we attempt direct substitution of \(x = -1 \), the function takes on the form of \(\frac{0}{0}\). Thus, we should check if \lh's rule can be applied. 
    Let \(f \defeq \sin\, (x^3 + 1)\), and \( g \defeq x^2 - 1\).\ \(f\) is a sinusoidal function, and \(g\) is a polynomial, so both of these are differentiable. Then, \(g' = 2x \neq 0\) on 
    the open interval \((-2, -\frac{1}{2})\), which contains \(x = -1\). Finally, we know that the last condition is satisfied because of our first observation. That means \lh's rule 
    can be applied. Thus,\begin{align*}
        \lim_{x\to -1} \frac{\sin \,(x^3 + 1)}{x^2 -1 } &= \lim_{x \to -1} \frac{\cos \, (x^3 + 1) \cdot 3x^2}{2x},\\ 
        \intertext{applying the chain rule to the numerator. This limit can then be evaluated by traditional means, as follows:}
        &= \lim_{x \to -1} \frac{\cos \, (x^3 + 1) \cdot \cancelto{3x}{3x^2}}{\cancelto{2}{2x}} = \frac{3}{2} \lim_{x\to -1} x \cos\, (x^3 + 1) = -\frac{3}{2}. 
    \end{align*}
    Therefore, the limit of \(\dfrac{\sin \,(x^3 + 1)}{x^2 -1 }\) as \(x\) approaches \(-1\) is \(-\dfrac{3}{2}\). 
\end{example} 
Once we invoke \lh's rule once, we can continue solving the limit as usual, using the same rules that we've learned from previous calculus classes, as we did here. However, it's also 
possible that once we \textit{do} proceed with this, we stumble across an indeterminate form \textit{again}, and we need to use \lh's rule once more. It's important to be 
careful in checking whether or not the conditions to the theorem apply.\par 
Before we proceed with another example, let's clarify what exactly an \textbf{indeterminate form} is, and why we call them as such. We'll also see that we can transform certain indeterminate forms 
into one that allows us to use \lh's rule.
\begin{center}
    \colorbox{Periwinkle!30}{\begin{minipage}{0.97\textwidth}
        \textbf{\scshape{Adventures in Sir Clark's Class.}} Say we don't know anything about \lh's rule, and we also don't know what to do when a limit evaluates to \(\frac{0}{0}\). So, we go about our lives solving things 
        with the limit laws, when we stumble across our favorite limit from 31.1:\[
            \lim_{x\to 0} \frac{\sin x}{x}.\quad\text{Not knowing any better, we use direct substitution and evaluate this to be}~\lim_{x\to 0} \frac{\sin x}{x} = \frac{0}{0}.
        \] Great! But then we stumble across another, similar looking limit:\[
            \lim_{x\to 0} \frac{\sin 2x}{x}.\quad\text{Again, clueless, we use direct substitution and conclude that this is also}~ \lim_{x\to 0} \frac{\sin 2x}{x} = \frac{0}{0}.
        \] But then, we attend Sir Clark's class. He then tells us 
        that the limit of \(\frac{\sin x}{x}\) as \(x\) approaches 0 is 1, and with some careful manipulation, he also shows that the limit of \(\frac{\sin 2x}{x}\) as \(x\) approaches 1
        is actually \(\frac{1}{2}\). Does this mean, by the transitive property of the equals sign, that\[
            \lim_{x\to 0} \frac{\sin x}{x} =\lim_{x\to 0} \frac{\sin 2x}{x} = \frac{0}{0} = \frac{1}{2} = 1? 
        \] 
    \end{minipage}}
\end{center}
You can easily see how this can be extended to make any number be equal to anything, which is why we \textit{never} write \(= \frac{0}{0}\) when evaluating limits. We instead 
say that the limit takes on the \textbf{form} of \(\frac{0}{0}\), to avoid any confusion. We then call this form \textbf{indeterminate} because, as this example has shown, 
two limits taking on an identical indeterminate form tells us nothing about what the actual value of the limits are.\par
Note, however, that the fraction \(\frac{0}{0}\) isn't the only indeterminate form. The theorem above also specifies that both limits could also diverge to infinity. Here's an example of what that
would look like. 
\begin{example}
    Evaluate \(\displaystyle\lim_{x\to \infty}\frac{\ln x}{\sqrt{x^2 + 1}}\).\par\vspace{3pt} 
    Clearly, if we were to just evaluate this limit at face value, we would get the form \(\frac{\infty}{\infty}\). This is an indeterminate form, so we can check
    if \lh's rule applies. Letting \(f \defeq \ln x\) and \(g \defeq \sqrt{x^2 + 1}\), we can see that both these functions are differentiable, that they are nonzero 
    on the interval \((0, \infty)\), and that both limits diverge to positive infinity as \(x\) grows larger and larger. We can then use \lh's rule. Thus,\begin{align*}
        \lim_{x\to \infty}\frac{\ln x}{\sqrt{x^2 + 1}} &= \lim_{x\to \infty} \frac{1/x}{\frac{1}{2}{(x^2 + 1)}^{-1/2} \cdot 2x} = \lim_{x\to\infty} \biggl( \frac{1}{x} \cdot \frac{\sqrt{x^2 + 1}}{x} \biggr) = \lim_{x\to\infty} \frac{\sqrt{x^2 +1}}{x^2}.\\
        \intertext{This limit, however, \textit{still} has an indeterminate form, so we use \lh's rule again.\footnotemark}
        &= \lim_{x\to\infty} \frac{\frac{1}{2}{(x^2 + 1)}^{-1/2} \cdot \cancel{2x}}{\cancel{2x}} = \lim_{x\to\infty}\frac{1}{2\sqrt{x^2 +1}} = 0. 
    \end{align*}
    Therefore, the limit of \(\dfrac{\ln x}{\sqrt{x^2 + 1}}\) as \(x\) approaches \(\infty\) is 0. 
\end{example}
\footnotetext{We omit the usual checks here, but it's straightforward to verify for yourself that the conditions are met.}
We see here the aforementioned case wherein we can repeatedly apply \lh's rule to the limit \textit{if} it takes on an indeterminate form again. 
We might be tempted to get a bit trigger-happy with this newfound tool, and try to find places where we can apply it, but it's important to always check 
if the conditions are met and that no other possible manipulation to make the limit more digestible is possible, because there are situations in which haphazardly applying \lh's rule can fail, and 
we might discover that there was a much easier solution available all along.
\begin{example}
    Evaluate \(\displaystyle\lim_{x\to -\infty}\frac{x + \sin x}{2x}\).\par\vspace{2pt}
    If we were to blindly attempt \lh's rule here (because direct substitution yields \(\frac{-\infty}{-\infty}\)), we would then get\begin{align*}
        \lim_{x\to -\infty}\frac{x + \sin x}{2x} &= \lim_{x\to -\infty} \frac{1 + \cos x}{2}, \\
        \intertext{which does not exist. This is then in violation of the last part of \lh's rule, which stipulates that the limit after differentiating 
        the numerator and the denominator must exist for the theorem to apply. But, notice that}
        \lim_{x\to -\infty}\frac{x + \sin x}{2x} &= \lim_{x\to -\infty} \biggl( \frac{x}{2x} + \frac{\sin x}{2x}\biggr) = \lim_{x\to -\infty} \biggl( \frac{1}{2} + \frac{\sin x}{2x}\biggr) \\
        &= \frac{1}{2} + \lim_{x\to -\infty} \frac{\sin x}{2x}. \\ 
        \intertext{We can solve this limit by the squeeze theorem (check this for yourself!), which evaluates to 0, so}
        &= \frac{1}{2} + 0 = \frac{1}{2}. 
    \end{align*} 
    Therefore, the limit of \(\dfrac{x + \sin x}{2x}\) as \(x\) approaches \(-\infty\) is \(\dfrac{1}{2}\).
\end{example}
\section{Improper integrals}
