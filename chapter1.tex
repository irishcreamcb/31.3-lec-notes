\chapter{Calculus I Addendums}
I've labelled this section as such because it really feels like these topics should have been covered in MATH 31.1 and 31.2, seeing as they're focused on
limits, derivatives, and integrals---all the key aspects of basic calculus---but they're just tacked on here for some reason. Regardless, here they are.
Section 1 focuses on \textbf{solving limits by using derivatives}, and section 2 focuses on \textbf{integrals of functions on infinite intervals}, \textit{or}
\textbf{functions that diverge to infinity on closed intervals.}  

\section{Indeterminate forms and \lh's rule} 
Recall the following limit from elementary calculus:\[
    \lim_{x\to 0 } \frac{\sin x}{x} = 1. 
\] You might remember that this limit was proved using arc lengths of a circle, or maybe the squeeze theorem. Either way, it's clear that we can't use
the simpler methods of solving a limit (like the limit laws) to find this one. Direct substitution yields the fraction \(\frac{0}{0}\), which isn't very helpful.\par 
Now, imagine you're Johann Bernoulli looking at this equation in the 17th century and you notice a few things:\begin{enumerate}
    \item that \(\sin x\) and \(x\) are both differentiable, with \(
        \dfrac{\der }{\der x} \sin x = \cos x \) and \( \dfrac{\der }{\der x} x = 1 
    \) respectively,
    \item that both \(f' = \cos x\) and \(g' = 1\) are continuous throughout the real numbers, and 
    \item the limit of \(\dfrac{\cos x}{1}\) as \(x\) approaches \(1\) is, trivially, 1. 
\end{enumerate} 
Huh! Isn't that interesting! Being the mathematician that you are, you immediately think of whether or not this can be generalized to include a broader range 
of functions, and not just relatively simple ones like \(\sin x\) and \(x\).\par 
You decide to investigate the fact that both functions were equal to 0 when evaluated at \(x = a\), so you set up two functions \(f\) and \(g\), where \(f(a) = g(a) = 0\). 
For good measure (and for our convenience), you assume that both their derivatives are continuous and that \(g'(x) \neq 0\). Then, you try evaluating the limit of their quotient, as such:\begin{align*}
    \lim_{x\to a} \frac{f(x)}{g(x)} &= \lim_{x\to a} \frac{f(x) - f(a)}{g(x) - g(a)},~\text{since \(f(a) = g(a) = 0\).}\\
    \intertext{Then, we can divide the numerator and denominator by the same value, \(x - a\), without changing the overall value:} 
    &= \lim_{x \to a} \frac{\dfrac{f(x) - f(a)}{x-a}}{\dfrac{g(x) - g(a)}{x - a}} = \frac{\displaystyle{} \lim_{x\to a} \frac{f(x) - f(a)}{x-a}}{\displaystyle{} \lim_{x\to a} \frac{g(x) - g(a)}{x-a}}\\ 
    \intertext{Wait a minute. That looks familiar! These limits on the numerator and the denominator are precisely \(f'(a)\) and \(g'(a)\)!}
    &= \frac{f'(a)}{g'(a)} = \lim_{x \to a} \frac{f'(x)}{g'(x)},~\text{since \(f'\) and \(g'\) are continuous.} 
\end{align*}
Of course, you being one of the smartest mathematicians of all time, you don't stop there. You generalize it even further to go beyond the rather restrictive conditions 
we placed on ourselves for this example to work as nicely as it did, armed with all the tools of your calculus-addled brain.\footnote{Proving the very generalized and 
extended \lh's rule involves a lot of calculus that we haven't learned yet. I might include a proof of it in the latter pages or something, but there are \textit{a lot} of cases.} 
Satisfied with yourself, but desperate for some money, you sell it to a colleague, who promptly publishes it, and the theorem is now forever named after him and not after you. 
Oh well. At least you have about a dozen generations left to solidify your family name's fame.\par 
We now introduced the generalized theorem to solve limits which take on the form \(\frac{0}{0}\) or \(\frac{\pm \infty}{\pm \infty}\) when direct substitution is attempted, named after
17th century mathematician Guillaume de \lh.
\begin{theo}[\lh's rule]
Suppose that\begin{itemize}
    \item two functions \(f\) and \(g\) are differentiable;
    \item \(g'(x) \neq 0 \) on some open interval \(I\) that contains a constant \(a\), except
possibly \(g'(a) = 0\); and 
    \item \(\lim_{x\to a} f(x) = \lim_{x\to a} g(x) = 0\), or both limits diverge to either positive or negative infinity.
\end{itemize} Then,\[
    \lim_{x\to a} \frac{f(x)}{g(x)} = \lim_{x\to a} \frac{f'(x)}{g'(x)}.
\]
\end{theo}
Note that the constant \(a\) that \(x\) approaches in this limit may be either from the right or the left (i.e.\ \(a^+\) or \(a^-\)), or could also be as \(x\) 
approaches either positive or negative infinity.\par 
\begin{example}
    Evaluate \(\displaystyle{} \lim_{x\to -1} \frac{\sin \,(x^3 + 1)}{x^2 -1 }\).\par 
    Notice here that when we attempt direct substitution of \(x = -1 \), the function takes on the form of \(\frac{0}{0}\). Thus, we should check if \lh's rule can be applied. 
    Let \(f \defeq \sin\, (x^3 + 1)\), and \( g \defeq x^2 - 1\).\ \(f\) is a sinusoidal function, and \(g\) is a polynomial, so both of these are differentiable. Then, \(g' = 2x \neq 0\) on 
    the open interval \((-2, -\frac{1}{2})\), which contains \(x = -1\). Finally, we know that the last condition is satisfied because of our first observation. That means \lh's rule 
    can be applied. Thus,\begin{align*}
        \lim_{x\to -1} \frac{\sin \,(x^3 + 1)}{x^2 -1 } &= \lim_{x \to -1} \frac{\cos \, (x^3 + 1) \cdot 3x^2}{2x},\\ 
        \intertext{applying the chain rule to the numerator. This limit can then be evaluated by traditional means, as follows:}
        &= \lim_{x \to -1} \frac{\cos \, (x^3 + 1) \cdot \cancelto{3x}{3x^2}}{\cancelto{2}{2x}} = \frac{3}{2} \lim_{x\to -1} x \cos\, (x^3 + 1) = -\frac{3}{2}. 
    \end{align*}
    Therefore, the limit of \(\dfrac{\sin \,(x^3 + 1)}{x^2 -1 }\) as \(x\) approaches \(-1\) is \(-\dfrac{3}{2}\). 
\end{example} 
Once we invoke \lh's rule once, we can continue solving the limit as usual, using the same rules that we've learned from previous calculus classes, as we did here. However, it's also 
possible that once we \textit{do} proceed with this, we stumble across an indeterminate form \textit{again}, and we need to use \lh's rule once more. It's important to be 
careful in checking whether or not the conditions to the theorem apply.\par 
Before we proceed with another example, let's take some time to clarify what exactly an indeterminate form is, and why we call them as such. Later, we'll also see that we can transform certain indeterminate forms 
into ones that allow us to use \lh's rule even when the original expression isn't a quotient.
\begin{center}
    \colorbox{CornflowerBlue!20}{\begin{minipage}{0.97\textwidth}
        Say we don't know anything about \lh's rule, and we also don't know that the 
    \end{minipage}}
\end{center}
\section{Improper integrals}
